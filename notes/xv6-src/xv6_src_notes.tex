\documentclass{article}

\usepackage{xv6}

%%% Title
\title{Notes about xv6}
\author{Yuting Wang}
\date{April 1st, 2025}



%%% Main body
\begin{document}

\maketitle

\section{Introduction}

This article contains information about building and debugging xv6,
and explanations of the source code of xv6 (x86 version, not
RISC-V). The information is gathered online and compiled with my own
understanding. The main references are

\begin{itemize}
\item xv6 book and source code sheet;
\item xv6-annotated \url{https://github.com/palladian1/xv6-annotated/tree/main};
\item Yale CS422 \url{https://flint.cs.yale.edu/cs422/assignments/index.html}
\item x86 instruction reference \url{https://www.felixcloutier.com/x86/}
\end{itemize}

\section{Compile and Debug}

To compile xv6 source in linux, run \code{make}. To debug the source
code in \code{qemu} with remote debugging, you need to open two
terminals in the source code directory~\cite{xv6-debug}:

\begin{itemize}
\item In one terminal, run \code{make qemu-nox-gdb} (No X windows,
  with GNU debugger);

\item In another terminal, first run \code{gdb kernel} to load the
  symbol table for kernel. In \code{gdb}, enter \code{target remote
    localhost:26000} (26000 is the port qemu provides for remote
  debugging; it depends on the result of the first command).
\end{itemize}

\section{Layout of Physical Memory}

The first \code{1MB} of physical memory is used by the earliest
PC. Its layout can be found here~\cite{yale-cs422-as1}.

\section{Bootloader}

Bootloader consists of \code{bootasm.S} and \code{bootmain.c}. It is
compiled separately from the kernel, resulting in a flat binary file
\code{bootblock} (not in ELF format). This file reside in the first
sector (512bytes) in the hard disk, and is loaded into address
\code{[0x7c00 - 0x7dff]} by the BIOS. After initializing the hardware
and this loading, the BIOS jumps to \code{0x7c00} to start executing
the boot loader.

To inspect the source code of \code{bootblok}, run the following command~\cite{disasm-flat}:

\begin{tabbing}
  \qquad\=\kill
  \> \code{objdump -D -b binary -m i386 bootblock}
\end{tabbing}

\section{Real-Mode}

The boot loader start running in real-mode with 16-bit registers and
values. In real-mode logical addresses \code{segment:offset} are used
to compute linear address \code{segment * 16 + offset}. Switching to
protected mode make use of GDT (global description table) for
computing logical addresses.

A segment selector \code{segment} is 16-bit long. Its highest 13-bit
is used for indexing into GDT (which contains $8192 = 2^{13}$
entries)~\cite{jasoncc-mem}. Each GDT entry is $8$-byte long,
containing $32$-bit base address, $20$-bit limit (with per-byte and
per-page granularity) and $12$-bit flags.

%%% Bibtex
\bibliographystyle{plain}
\bibliography{refs}

\end{document}

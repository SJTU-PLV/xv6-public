\documentclass{article}

\usepackage{xv6}

%%% Title
\title{Notes about xv6}
\author{Yuting Wang}
\date{April 1st, 2025}



%%% Main body
\begin{document}

\maketitle

\section{Introduction}

This article contains information about building and debugging xv6,
and explanations of the source code of xv6 (x86 version, not
RISC-V). The information is gathered online and compiled with my own
understanding. The main references are

\begin{itemize}
\item xv6 book and source code sheet~\cite{xv6-x86};
\item xv6-annotated~\cite{xv6-annotated};
\item Yale CS422~\cite{yale-cs422};
\item x86 instruction reference~\cite{x86-reference}.
\end{itemize}

\section{Compile and Debug}

To compile xv6 source in linux, run \code{make}. To debug the source
code in \code{qemu} with remote debugging, you need to open two
terminals in the source code directory~\cite{xv6-debug}:

\begin{itemize}
\item In one terminal, run \code{make qemu-nox-gdb} (No X windows,
  with GNU debugger);

\item In another terminal, first run \code{gdb kernel} to load the
  symbol table for kernel. In \code{gdb}, enter \code{target remote
    localhost:26000} (26000 is the port qemu provides for remote
  debugging; it depends on the result of the first command).
\end{itemize}

\section{Layout of Physical Memory}

The first \code{1MB} of physical memory is used by the earliest
PC. Its layout can be found here~\cite{yale-cs422-as1}.

\section{Bootloader}

Bootloader consists of \code{bootasm.S} and \code{bootmain.c}. It is
compiled separately from the kernel, resulting in a flat binary file
\code{bootblock} (not in ELF format). This file reside in the first
sector (512bytes) in the hard disk, and is loaded into address
\code{[0x7c00 - 0x7dff]} by the BIOS. After initializing the hardware
and this loading, the BIOS jumps to \code{0x7c00} to start executing
the boot loader.

To inspect the source code of \code{bootblok}, run the following command~\cite{disasm-flat}:

\begin{tabbing}
  \qquad\=\kill
  \> \code{objdump -D -b binary -m i386 bootblock}
\end{tabbing}

\section{Real-Mode to Protected Mode}

The boot loader start running in real-mode with 16-bit registers and
values. In real-mode a logical address \code{segment:offset} is used
to compute linear address \code{segment * 16 + offset} where
\code{segment} is called a \emph{segment selector}.

The first job of \code{bootasm.S} is to switch from the real-mode to
protected mode. In protected mode, the translation of logical
addresses into linear addresses are different. It make use of GDT
(global description table). A GDT contains $8192$ entries of
\emph{segment descriptors}, each of which is 8-byte long. Each
descriptor contains a $32$-bit base address, a $20$-bit limit (with
per-byte and per-page granularity) and a $12$-bit flags. The macros
\code{SEG\_ASM} and \code{SEG\_NULLASM} for creating those descriptors
are defined in \code{asm.h} which I added comments to describe their
mechanisms.

In the protected mode, segment selector registers are used to store
selectors for indexing into the GDT (\code{\%cs} for code segment,
\code{\%ds} for data segment, etc). A selector is 16-bit long with the
following format:
%
\begin{verbatim}
  15      3  2   0 
  +--------+--+---+  INDEX: select one of the 8192 descriptors
  | INDEX  |TI|RPL|  TI: Table Indicator {0=GDT, 1=LDT}
  +--------+--+---+  RPL: Requested Privilege Level (Ring 0-3)
\end{verbatim}
%
Its highest 13-bit is used for indexing into GDT (which contains
$8192 = 2^{13}$ entries)~\cite{jasoncc-mem}. \code{RPL} (2-bits)
indicates the privilege level requested (0-3 ring levels).

In the protected mode, the translation from a logical address
\code{segment:offset} to a linear address is as follows:
%
\begin{itemize}
\item Extract the index \code{i} from the selector \code{segment}, use
  it to find \code{i}-th segment descriptor in GDT. Let us call it
  \code{sdi}.

\item Extract \code{base} and \code{limit} from \code{sdi}.

\item The linear address is \code{base + offset}. If it exceeds
  \code{limit}, then this address is invalid.
\end{itemize}
%
Therefore, the job of \code{bootasm.S} is to 1) set up a GDT containing
necessary descriptors, and 2) initialize the segment selector
registers (\code{\%cs}, \code{\%ds}, \code{\%es}, \code{\%ss},
\code{\%fs}, and \code{\%gs}) with appropriate indices into the GDT.
%
Step 1) is completed by the following instruction:
%
\begin{verbatim}
  lgdt    gdtdesc
\end{verbatim}
%
The GDT contains three entries, the first entry is NULL, the second is
the code segment, the third is the data segment. The base addresses
are all set to $0$ for an identity mapping from logical addresses to
linear addresses (and to the physical addresses).
%
Step 2) is completed by the following instructions:
%
\begin{verbatim}
  ljmp    $(SEG_KCODE<<3), $start32

.code32  # Tell assembler to generate 32-bit code now.
start32:
  # Set up the protected-mode data segment registers
  movw    $(SEG_KDATA<<3), %ax    # Our data segment selector
  movw    %ax, %ds                # -> DS: Data Segment
  movw    %ax, %es                # -> ES: Extra Segment
  movw    %ax, %ss                # -> SS: Stack Segment
  movw    $0, %ax                 # Zero segments not ready for use
  movw    %ax, %fs                # -> FS
  movw    %ax, %gs                # -> GS
\end{verbatim}
%
Note that \code{SEG\_KCODE = 1} is used to index into the first entry
(code segment) in the GDT. The first argument of \code{ljmp} is
\code{\$(SEG\_KCODE<<3)} which is set to the new value of
\code{\%cs}. The left shift is to move \code{SEG\_KCODE} into its
appropriate position (bypassing \code{TI} and \code{RPL}). Same
processing is required for indexing into the data segment with
\code{SEG\_KDATA=2}. The unused registers \code{\%fs} and \code{\%gs}
are set with index $0$ which points into the null segment.

The last step of \code{bootasm.S} is to setup the stack and invoke the
C part of the boot loader. The following instructions perform this
task:
%
\begin{verbatim}
  movl    $start, %esp
  call    bootmain
\end{verbatim}
%
The immediate value \code{\$start} is the beginning of
\code{bootasm.S} (or the binary \code{bootblock}). This value is set
to \code{0x7c00} by Makefile for linking \code{bootasm.S} and
\code{bootmain.c} into \code{bootblock}. That is, it is the beginning
of the physical address where the boot loader resides at run time. By
the memory layout~\cite{yale-cs422-as1}, the address range \code{[0 -
  0x7c00]} is not used. Since stack grows towards lower addresses,
\code{\%esp} is set to the topmost address of this unused region.


\section{Possible Errors in the Resources}

\begin{itemize}
\item At \url{https://www.felixcloutier.com/x86/lgdt:lidt}, it is said
  that an GDT entry is 6 bytes, which contradicts with the code that
  assume an GDT entry is 8 bytes.

\item At
  \url{https://github.com/palladian1/xv6-annotated/blob/main/boot.md},
  it is said that ``a logical address consists of a 20-bit segment
  selector and a 12-bit offset''. However, the segment selector should
  be 16-bit long while the offset should be 32-bit long. 

\end{itemize}


%%% Bibtex
\bibliographystyle{plain}
\bibliography{refs}

\end{document}

\documentclass{article}

\usepackage{xv6}

%%% Title
\title{Notes about xv6}
\author{Yuting Wang}
\date{April 1st, 2025}



%%% Main body
\begin{document}

\maketitle

\section{Introduction}

This article contains information about building and debugging xv6,
and explanations of the source code of xv6. The information is
gathered online and compiled with my own understanding.

\section{Compile and Debug}

To compile xv6 source in linux, run \code{make}. To debug the source
code in \code{qemu} with remote debugging, you need to open two
terminals in the source code directory~\cite{xv6-debug}:

\begin{itemize}
\item In one terminal, run \code{make qemu-nox-gdb} (No X windows,
  with GNU debugger);

\item In another terminal, first run \code{gdb kernel} to load the
  symbol table for kernel. In \code{gdb}, enter \code{target remote
    localhost:26000} (26000 is the port qemu provides for remote
  debugging; it depends on the result of the first command).
\end{itemize}

\section{Layout of Physical Memory}

The first \code{1MB} of physical memory is used by the earliest
PC. Its layout can be found here~\cite{yale-cs422-as1}.

\section{Bootloader}

Bootloader consists of \code{bootasm.S} and \code{bootmain.c}. It is
compiled separately from the kernel, resulting in a flat binary file
\code{bootblock} (not in ELF format). This file reside in the first
sector (512bytes) in the hard disk, and is loaded into address
\code{[0x7c00 - 0x7dff]} by the BIOS. After initializing the hardware
and this loading, the BIOS jumps to \code{0x7c00} to start executing
the boot loader. To inspect the 

%%% Bibtex
\bibliographystyle{plain}
\bibliography{refs}

\end{document}
